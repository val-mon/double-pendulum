\documentclass[12pt,a4paper]{article}
\usepackage[utf8]{inputenc}
\usepackage[T1]{fontenc}
\usepackage{amsmath,amssymb}
\usepackage{geometry}
\usepackage{enumitem}
\usepackage{physics}
\usepackage{graphicx}
\usepackage{xcolor}
\usepackage{array}

\geometry{margin=2.5cm}

\title{\textbf{Double Pendule} \\ \large Approche Lagrangienne (Energies)}
\author{Valentin Monod}
\date{}

\begin{document}

\maketitle

\section{Configuration du Système}

\subsection{Paramètres}

\begin{itemize}[leftmargin=*]
    \item Masse du premier pendule : $m_1$ [kg]
    \item Masse du deuxième pendule : $m_2$ [kg]
    \item Longueur de la première tige : $L_1$ [m]
    \item Longueur de la deuxième tige : $L_2$ [m]
    \item Accélération gravitationnelle : $g = 9{,}81$ [m/s²]
\end{itemize}

\subsection{Coordonnées Généralisées}

\begin{itemize}[leftmargin=*]
    \item $\theta_1$ : angle du premier pendule avec la verticale (positif dans le sens horaire)
    \item $\theta_2$ : angle du deuxième pendule avec la verticale (positif dans le sens horaire)
    \item $\Delta\theta = \theta_1 - \theta_2$ : angle relatif entre les deux tiges
\end{itemize}

\subsection{Conversion Angles $\rightarrow$ Positions Cartésiennes}

Position de la masse $m_1$ :
\begin{align}
x_1 &= L_1 \sin(\theta_1) \\
y_1 &= -L_1 \cos(\theta_1)
\end{align}

Position de la masse $m_2$ :
\begin{align}
x_2 &= L_1 \sin(\theta_1) + L_2 \sin(\theta_2) \\
y_2 &= -L_1 \cos(\theta_1) - L_2 \cos(\theta_2)
\end{align}

\section{Vitesses}

\subsection{Dérivées des Positions}

Vitesse de $m_1$ :
\begin{align}
\dot{x}_1 &= L_1 \dot{\theta}_1 \cos(\theta_1) \\
\dot{y}_1 &= L_1 \dot{\theta}_1 \sin(\theta_1)
\end{align}

Vitesse de $m_2$ :
\begin{align}
\dot{x}_2 &= L_1 \dot{\theta}_1 \cos(\theta_1) + L_2 \dot{\theta}_2 \cos(\theta_2) \\
\dot{y}_2 &= L_1 \dot{\theta}_1 \sin(\theta_1) + L_2 \dot{\theta}_2 \sin(\theta_2)
\end{align}

\subsection{Normes des Vitesses}

Vitesse au carré de $m_1$ :
\begin{equation}
v_1^2 = \dot{x}_1^2 + \dot{y}_1^2 = L_1^2 \dot{\theta}_1^2
\end{equation}

Vitesse au carré de $m_2$ :
\begin{equation}
v_2^2 = L_1^2 \dot{\theta}_1^2 + L_2^2 \dot{\theta}_2^2 + 2L_1L_2 \dot{\theta}_1\dot{\theta}_2 \cos(\Delta\theta)
\end{equation}

\section{Énergies}

\subsection{Énergie Cinétique}

Énergie cinétique de $m_1$ :
\begin{equation}
T_1 = \frac{1}{2}m_1 v_1^2 = \frac{1}{2}m_1 L_1^2 \dot{\theta}_1^2
\end{equation}

Énergie cinétique de $m_2$ :
\begin{equation}
T_2 = \frac{1}{2}m_2 v_2^2 = \frac{1}{2}m_2 \left[ L_1^2 \dot{\theta}_1^2 + L_2^2 \dot{\theta}_2^2 + 2L_1L_2 \dot{\theta}_1\dot{\theta}_2 \cos(\Delta\theta) \right]
\end{equation}

Énergie cinétique totale :
\begin{equation}
T = \frac{1}{2}(m_1 + m_2)L_1^2 \dot{\theta}_1^2 + \frac{1}{2}m_2 L_2^2 \dot{\theta}_2^2 + m_2 L_1 L_2 \dot{\theta}_1\dot{\theta}_2 \cos(\Delta\theta)
\end{equation}

\subsection{Énergie Potentielle}

Référence : $y = 0$ au niveau du pivot (vers le bas est négatif).

Énergie potentielle de $m_1$ :
\begin{equation}
V_1 = m_1 g y_1 = -m_1 g L_1 \cos(\theta_1)
\end{equation}

Énergie potentielle de $m_2$ :
\begin{equation}
V_2 = m_2 g y_2 = -m_2 g [L_1 \cos(\theta_1) + L_2 \cos(\theta_2)]
\end{equation}

Énergie potentielle totale :
\begin{equation}
V = -(m_1 + m_2)g L_1 \cos(\theta_1) - m_2 g L_2 \cos(\theta_2)
\end{equation}

\section{Lagrangien}

Le Lagrangien est défini comme :
\begin{equation}
\mathcal{L} = T - V
\end{equation}

\begin{align}
\mathcal{L}(\theta_1, \theta_2, \dot{\theta}_1, \dot{\theta}_2) = & \frac{1}{2}(m_1 + m_2)L_1^2 \dot{\theta}_1^2 + \frac{1}{2}m_2 L_2^2 \dot{\theta}_2^2 \nonumber \\
& + m_2 L_1 L_2 \dot{\theta}_1 \dot{\theta}_2 \cos(\Delta\theta) \nonumber \\
& + (m_1 + m_2)g L_1 \cos(\theta_1) + m_2 g L_2 \cos(\theta_2)
\end{align}

\section{Équations d'Euler-Lagrange}

Les équations d'Euler-Lagrange donnent :
\begin{equation}
\frac{d}{dt}\left(\frac{\partial \mathcal{L}}{\partial \dot{\theta}_i}\right) - \frac{\partial \mathcal{L}}{\partial \theta_i} = 0, \quad i = 1, 2
\end{equation}

\subsection{Équation pour $\theta_1$}

\begin{equation}
(m_1 + m_2)L_1 \ddot{\theta}_1 + m_2 L_2 \ddot{\theta}_2 \cos(\Delta\theta) + m_2 L_2 \dot{\theta}_2^2 \sin(\Delta\theta) + (m_1 + m_2)g \sin(\theta_1) = 0
\end{equation}

\subsection{Équation pour $\theta_2$}

\begin{equation}
L_2 \ddot{\theta}_2 + L_1 \ddot{\theta}_1 \cos(\Delta\theta) - L_1 \dot{\theta}_1^2 \sin(\Delta\theta) + g \sin(\theta_2) = 0
\end{equation}

\section{Résolution pour les Accélérations}

Système linéaire de 2 équations à 2 inconnues $(\ddot{\theta}_1, \ddot{\theta}_2)$ :
\begin{equation}
\begin{cases}
(m_1 + m_2)L_1 \ddot{\theta}_1 + m_2 L_2 \cos(\Delta\theta) \ddot{\theta}_2 = -m_2 L_2 \dot{\theta}_2^2 \sin(\Delta\theta) - (m_1 + m_2)g \sin(\theta_1) \\
L_1 \cos(\Delta\theta) \ddot{\theta}_1 + L_2 \ddot{\theta}_2 = L_1 \dot{\theta}_1^2 \sin(\Delta\theta) - g \sin(\theta_2)
\end{cases}
\end{equation}

\subsection{Forme Matricielle}

\begin{equation}
\begin{bmatrix}
(m_1 + m_2)L_1 & m_2 L_2 \cos(\Delta\theta) \\
L_1 \cos(\Delta\theta) & L_2
\end{bmatrix}
\begin{bmatrix}
\ddot{\theta}_1 \\
\ddot{\theta}_2
\end{bmatrix}
=
\begin{bmatrix}
-m_2 L_2 \dot{\theta}_2^2 \sin(\Delta\theta) - (m_1 + m_2)g \sin(\theta_1) \\
L_1 \dot{\theta}_1^2 \sin(\Delta\theta) - g \sin(\theta_2)
\end{bmatrix}
\end{equation}

\subsection{Déterminant}

\begin{equation}
\Delta = m_1 + m_2 \sin^2(\Delta\theta)
\end{equation}

\subsection{Solutions Explicites}

Par la règle de Cramer :

\subsubsection{Accélération angulaire 1 :}

\begin{align}
\ddot{\theta}_1 = \frac{1}{L_1\Delta} \Big[ & -m_2 L_1 \dot{\theta}_1^2 \sin(\Delta\theta) \cos(\Delta\theta) - m_2 L_2 \dot{\theta}_2^2 \sin(\Delta\theta) \nonumber \\
& + m_2 g \sin(\theta_2) \cos(\Delta\theta) - (m_1 + m_2)g \sin(\theta_1) \Big]
\end{align}

\subsubsection{Accélération angulaire 2 :}

\begin{align}
\ddot{\theta}_2 = \frac{(m_1 + m_2)}{L_2\Delta} \Big[ & L_1 \dot{\theta}_1^2 \sin(\Delta\theta) + \frac{m_2}{m_1+m_2} L_2 \dot{\theta}_2^2 \sin(\Delta\theta) \cos(\Delta\theta) \nonumber \\
& + g \sin(\theta_1) \cos(\Delta\theta) - g \sin(\theta_2) \Big]
\end{align}

\section{Formules Finales pour la Simulation}

\subsection{Variables d'État}

\begin{equation}
\text{État}(t) = \begin{bmatrix}
\theta_1(t) \\
\theta_2(t) \\
\omega_1(t) \\
\omega_2(t)
\end{bmatrix}
\quad \text{où } \omega_1 = \dot{\theta}_1, \quad \omega_2 = \dot{\theta}_2
\end{equation}

\subsection{Équations Différentielles}

\begin{align}
\frac{d\theta_1}{dt} &= \omega_1 \\
\frac{d\theta_2}{dt} &= \omega_2 \\
\frac{d\omega_1}{dt} &= \ddot{\theta}_1(\theta_1, \theta_2, \omega_1, \omega_2) \\
\frac{d\omega_2}{dt} &= \ddot{\theta}_2(\theta_1, \theta_2, \omega_1, \omega_2)
\end{align}

avec :
\begin{equation}
\Delta\theta = \theta_1 - \theta_2, \quad \Delta = m_1 + m_2 \sin^2(\Delta\theta)
\end{equation}

\begin{align}
\ddot{\theta}_1 = \frac{1}{L_1\Delta} \Big[ & -m_2 L_1 \omega_1^2 \sin(\Delta\theta) \cos(\Delta\theta) - m_2 L_2 \omega_2^2 \sin(\Delta\theta) \nonumber \\
& + m_2 g \sin(\theta_2) \cos(\Delta\theta) - (m_1 + m_2)g \sin(\theta_1) \Big]
\end{align}

\begin{align}
\ddot{\theta}_2 = \frac{1}{L_2\Delta} \Big[ & (m_1+m_2) L_1 \omega_1^2 \sin(\Delta\theta) + m_2 L_2 \omega_2^2 \sin(\Delta\theta) \cos(\Delta\theta) \nonumber \\
& + (m_1+m_2) g \sin(\theta_1) \cos(\Delta\theta) - (m_1+m_2) g \sin(\theta_2) \Big]
\end{align}

\section{Conservation de l'Énergie}

L'énergie mécanique totale doit rester constante :
\begin{equation}
E_{\text{mec}} = T + V = \text{constante}
\end{equation}

\begin{align}
E_{\text{mec}} = & \frac{1}{2}(m_1 + m_2)L_1^2 \omega_1^2 + \frac{1}{2}m_2 L_2^2 \omega_2^2 + m_2 L_1 L_2 \omega_1 \omega_2 \cos(\Delta\theta) \nonumber \\
& - (m_1 + m_2)g L_1 \cos(\theta_1) - m_2 g L_2 \cos(\theta_2)
\end{align}

\section{Conversion pour la Visualisation}

Pour afficher les positions cartésiennes :
\begin{align}
x_1 &= L_1 \sin(\theta_1) \\
y_1 &= -L_1 \cos(\theta_1) \\
x_2 &= L_1 \sin(\theta_1) + L_2 \sin(\theta_2) \\
y_2 &= -L_1 \cos(\theta_1) - L_2 \cos(\theta_2)
\end{align}

\end{document}
